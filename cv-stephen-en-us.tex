% LaTeX Curriculum Vitae Template
%
% Copyright (C) 2004-2009 Jason Blevins <jrblevin@sdf.lonestar.org>
% http://jblevins.org/projects/cv-template/
%
% You may use use this document as a template to create your own CV
% and you may redistribute the source code freely. No attribution is
% required in any resulting documents. I do ask that you please leave
% this notice and the above URL in the source code if you choose to
% redistribute this file.

\documentclass[letterpaper]{article}

\usepackage{hyperref}
\usepackage{geometry}
\usepackage[utf8]{inputenc}

% Comment the following lines to use the default Computer Modern font
% instead of the Palatino font provided by the mathpazo package.
% Remove the 'osf' bit if you don't like the old style figures.
\usepackage[T1]{fontenc}
\usepackage[sc,osf]{mathpazo}

% Set your name here
\def\name{Stephen Pedrosa Eilert}

% Replace this with a link to your CV if you like, or set it empty
% (as in \def\footerlink{}) to remove the link in the footer:
\def\footerlink{http://github.com/outworlder/CV.git/}

% The following metadata will show up in the PDF properties
\hypersetup{
  colorlinks = true,
  urlcolor = black,
  pdfauthor = {\name},
  %pdfkeywords = {it, Computer Science},
  pdftitle = {\name: Curriculum Vitae},
  pdfsubject = {Curriculum Vitae},
  pdfpagemode = UseNone
}

\geometry{
  body={6.5in, 8.5in},
  left=1.0in,
  top=1.25in
}

% Customize page headers
\pagestyle{myheadings}
\markright{\name}
\thispagestyle{empty}

% Custom section fonts
\usepackage{sectsty}
\sectionfont{\rmfamily\mdseries\Large}
\subsectionfont{\rmfamily\mdseries\itshape\large}

% Other possible font commands include:
% \ttfamily for teletype,
% \sffamily for sans serif,
% \bfseries for bold,
% \scshape for small caps,
% \normalsize, \large, \Large, \LARGE sizes.

% Don't indent paragraphs.
\setlength\parindent{0em}

% Make lists without bullets
%% \renewenvironment{itemize}{
%%   \begin{list}{}{
%%     \setlength{\leftmargin}{1.5em}
%%   }
%% }{
%%   \end{list}
%% }

\begin{document}

\begin{flushright}

{\huge \name}

% Alternatively, print name centered and bold:
%\centerline{\huge \bf \name}

\vspace{0.20in}
\begin{minipage}{0.45\linewidth}
  \begin{tabular}{ll}
    Cellphone: & +1 (669) 226-7520 \\
    Email: & \href{mailto:contact@stepheneilert.com}{\tt contact@stepheneilert.com} \\
    Github: & \href{https://github.com/outworlder}{\tt github.com/outworlder}
  \end{tabular}
\end{minipage}
\end{flushright}

\section*{Summary}
  Sofware developer with 19 years of experience in a variety of projects for customers in the US and abroad. Experience with the full software development lifecycle, from requirements, design, development, testing, deployment, and system administration. Highly motivated by projects that challenge the technological status-quo. Somewhat paradoxically, likes to work closer to the machine, but also at the high levels of abstraction provided by functional languages. Follows with interest new technology developments, and will learn interesting technologies even when they are not directly relevant to the current work.

\section*{Education}

\begin{itemize}
  \item BSc. Computer Science, Universidade Federal do Ceará, 2008.
\end{itemize}

\section*{Professional Experience}

\begin{enumerate}
  \item
  \textbf{Hewlett Packard Enterprise, 2021 - Present} \\
  Distinguished Technologist. Architected the 'one-click' solution, which enables one-touch provisioning of complex Kubernetes clusters with their infrastructure (networking, databases, security) in a modular way, which works in AWS, GCP, and Azure. Responsible for the architecture of the Common Cloud Platform, used internally by multiple business units. Provides technical guidance for several teams in the company.
  \item
  \textbf{Aruba Networks, a Hewlett Packard Enterprise Company, 2015 - 2021} \\
  Systems Software Engineer - Devops. Joined the Aruba Devops organization to work on automation. Improved the existing scripts to completely remove manual provisioning steps and human decision-making. This turned an activity that required multiple people across two timezones and at least one month to complete into a single-person, 20-minute job. Created the 'ClusterDB' - a single source of truth for all automation activities. Won company awards for his role in solving the technical issues that enabled a successful deployment of the Aruba Central product in China.
  \item
  \textbf{Hewlett Packard Enterprise, 2015 - 2016} \\
  Systems Software Engineer. Joined the Openstack Neutron team, to work upstream. Delivered enhancements to Neutron to support VMWare ESX's distributed virtual switch (vDS). Fixed issues and was in charge of maintenance of the internal CI servers (Jenkins). Focus shifted to HP's Helion Openstack (HOS), backporting patches, fixing CI issues, and adding features and bugfixes to its Ansible-based installer.
  \item 
  \textbf{Hewlett-Packard Brazil, 2014 - 2015} \\
  Software engineer, working with an experienced team to develop a high-profile platform that handles printer monitoring and supplies. Technologies range from .NET desktop and web apps, printer communication protocols, and single-page web applications in Javascript/Backbone.js. Emphasis on code quality, with mandatory code reviews, high code test coverage and continuous integration.
  \item
  \textbf{Instituto Atlântico, 2013 - 2014} \\
  Software developer and tech lead on contract for Hewlett-Packard.
  \item
  \textbf{Guilda (founder) / Codeminer 42, 2012 - 2013 } \\
  Co-founded a company to develop Mobile (iOS and Android), C++ (with Qt), and Ruby on Rails applications for customers. Also provided consulting services and developed a cloud-based continuous integration system (currently discontinued) – one of the first in the market to use LXC containers. Acquired by Codeminer42.

  \item
  \textbf{DETRAN-CE, 2010 - 2012} \\
  Software Engineer. Web systems in Ruby on Rails, consuming data from PostgreSQL, MongoDB, and Cassandra. Development of a fingerprint recognition platform using libfprint, in both C and Python. Creation of a high-performance middleware application in C to broker communication between internal and external systems with the government's mainframe.
  \item
  \textbf{Instituto Atlântico, 2004 - 2009 } \\
  Developer and technical leader in several projects for a variety of customers. Among them, was the development of a new banking platform in Java (J2EE) to replace the aging, mainframe-based system. Also of note was an image-based (using an optical-flow algorithm), motion detection module for a proprietary mobile operating system. Spent some time in Gibraltar with a team to finish development, deployment of the customer's new platform in Ruby on Rails, as well as measuring performance and adding a caching layer (with Memcached) so that it would support an upcoming major sporting event.
  \item
  \textbf{Media System, 2003 - 2004} \\
  Trainee. .NET desktop and mobile application development.
\end{enumerate}

\bigskip

\section*{Spoken Languages}
\begin{itemize}
  \item English (Advanced)
  \item Portuguese (Native)
\end{itemize}

\section*{Certifications}
\begin{itemize}
  \item Sun Certified Java Programmer - SCJP 5.0
  \item Cambridge FSOL FCE - University of Cambridge
\end{itemize}

\section*{Skills}
\begin{itemize}
  \item Cloud: AWS, Google Cloud, Azure
  \item Containerization: Kubernetes, Docker, familiar with many CNCF projects
  \item Devops: Terraform, Ansible, Chef, Packer, Consul, Vault, Prometheus
  \item Web technologies: HTML5, CSS3, Javascript, React
  \item Oracle Technologies: (Java Platform including J2SE, JEE, and J2ME)
  \item Extensive Linux knowledge.
  \item Source Control: Git, Mercurial, SVN
  \item Programming Languages: Python, Go, C, C++, Java, C\#, Javascript, Ruby, Lua, Objective-C, Scheme
  \item Database Management Systems: PostgreSQL, MySQL
  \item NoSQL databases: Cassandra, Kafka, Redis, Memcachedb, Elasticsearch
  \item Mobile Development: some Android, heavy focus on iOS
  \item Microsoft Technologies: (.NET Framework, Win32 API, SQL Server, Team Foundation Server, PowerShell)
\end{itemize}


% Footer
\begin{center}
  \begin{footnotesize}
    Last updated: \today \\
    \href{\footerlink}{\texttt{\footerlink}}
  \end{footnotesize}
\end{center}

\end{document}
