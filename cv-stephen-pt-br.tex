% LaTeX Curriculum Vitae Template
%
% Copyright (C) 2004-2009 Jason Blevins <jrblevin@sdf.lonestar.org>
% http://jblevins.org/projects/cv-template/
%
% You may use use this document as a template to create your own CV
% and you may redistribute the source code freely. No attribution is
% required in any resulting documents. I do ask that you please leave
% this notice and the above URL in the source code if you choose to
% redistribute this file.

\documentclass[letterpaper]{article}

\usepackage{hyperref}
\usepackage{geometry}
\usepackage[utf8]{inputenc}

% Comment the following lines to use the default Computer Modern font
% instead of the Palatino font provided by the mathpazo package.
% Remove the 'osf' bit if you don't like the old style figures.
\usepackage[T1]{fontenc}
\usepackage[sc,osf]{mathpazo}

% Set your name here
\def\name{Stephen Pedrosa Eilert}

% Replace this with a link to your CV if you like, or set it empty
% (as in \def\footerlink{}) to remove the link in the footer:
\def\footerlink{http://github.com/spedrosa/CV.git/}

% The following metadata will show up in the PDF properties
\hypersetup{
  colorlinks = true,
  urlcolor = black,
  pdfauthor = {\name},
  %pdfkeywords = {economics, statistics, mathematics},
  pdftitle = {\name: Curriculum Vitae},
  pdfsubject = {Curriculum Vitae},
  pdfpagemode = UseNone
}

\geometry{
  body={6.5in, 8.5in},
  left=1.0in,
  top=1.25in
}

% Customize page headers
\pagestyle{myheadings}
\markright{\name}
\thispagestyle{empty}

% Custom section fonts
\usepackage{sectsty}
\sectionfont{\rmfamily\mdseries\Large}
\subsectionfont{\rmfamily\mdseries\itshape\large}

% Other possible font commands include:
% \ttfamily for teletype,
% \sffamily for sans serif,
% \bfseries for bold,
% \scshape for small caps,
% \normalsize, \large, \Large, \LARGE sizes.

% Don't indent paragraphs.
\setlength\parindent{0em}

% Make lists without bullets
%% \renewenvironment{itemize}{
%%   \begin{list}{}{
%%     \setlength{\leftmargin}{1.5em}
%%   }
%% }{
%%   \end{list}
%% }

\begin{document}

\begin{flushright}

{\huge \name}

% Alternatively, print name centered and bold:
%\centerline{\huge \bf \name}

\vspace{0.25in}
\begin{minipage}{0.45\linewidth}
  \begin{tabular}{ll}
    Celular: & 85 9996-6984 \\
    Telefone: & 85 3272-6747 \\
    Email: & \href{mailto:spedrosa@gmail.com}{\tt spedrosa@gmail.com} \\
    Página Web: & \href{http://blog.paleolithic-computing.com}{\tt http://blog.paleolithic-computing.com} \\
  \end{tabular}
\end{minipage}
\end{flushright}

\section*{Formação acadêmica}

\begin{itemize}
  \item Bacharelado em Ciências da Computação, Universidade Federal do Ceará, 2008.
\end{itemize}

\section*{Habilidades}
\begin{itemize}
  \item Tecnologias Microsoft (.NET Framework, API Win32, SQL Server)
  \item Tecnologias Oracle (Platforma Java including J2SE, J2EE and J2ME) e tecnologias
    de suporte como Hibernate, Spring, Struts.
    Oracle Database e Oracle Application Server.
  \item Unix (Linux, NetBSD, Solaris, MacOS X)
  \item Modelagem UML
  \item Linguagens de Programação: Ruby, Python, Lua, C, C++, C\#, Java, Objective-C, Scheme, Lisp, Clojure
  \item Metodologias de desenvolvimento como RUP e ágeis como SCRUM
  \item Ferramentas: Rational Rose, Rational Unified Modeler, Enterprise Architect, Microsoft
Visual Studio, Eclipse, Ant, Maven, Gradle, debuggers e profilers (específicos de cada linguagem e plataforma),
testes unitários e integração contínua (CruiseControl, Luntbuild, RSpec, autotest, TravisCI), Xcode
  \item Desenvolvimento Mobile (iOS, Android)
  \item Tecnologias em nuvem (ambientes da Amazon e Rackspace)
\end{itemize}


\section*{Certificações}
\begin{itemize}
  \item Sun Certified Java Programmer - SCJP 5.0
  \item Cambridge FSOL FCE - University of Cambridge
\end{itemize}

\section*{Vivência Internacional}
\begin{itemize}
  \item Gibraltar (Territôrio Britânico) - 4 meses (Projeto LVS)
\end{itemize}

\section*{Experiência profissional}

\begin{enumerate}
  \item 
  \textbf{Guilda (founder), 2012 - 2013 } \\
  Desenvolvimento de aplicações em Ruby on Rails, Mobile (iOS e Android) e C++ (QT), para clientes no Brasil e exterior, além de produtos próprios.
  \item
  \textbf{DETRAN-CE, 2010 - 2012} \\
  Desenvolvimento de diversos sistemas internos, utilizando Ruby on Rails. Plataforma de reconhecimento de impressões digitais com libfprint, Python e C. Criação de middleware de alta performance para integração dos sistemas internos com o mainframe do SERPRO. Utilização de metodologias ágeis.
  \item
  \textbf{Instituto Atlântico, 2004 - 2009 } \\
  Atuação como desenvolvedor e líder técnico em diversos projetos, utilizando plataformas Microsoft (C\#, .NET com SQL Server), Java (Oracle Application Server), bem como software para celular embarcado (C). 
  \item
  \textbf{Media System, 2003 - 2004} \\
  Estagiário, desenvolvedor. Desenvolvimento de aplicações em .NET para desktop e Mobile (Windows CE, Windows Mobile)
\end{enumerate}

\bigskip

% Footer
\begin{center}
  \begin{footnotesize}
    Última atualização: \today \\
    \href{\footerlink}{\texttt{\footerlink}}
  \end{footnotesize}
\end{center}

\end{document}
